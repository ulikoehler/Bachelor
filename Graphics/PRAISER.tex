\documentclass{article}
\usepackage{tikz,amsmath,siunitx}
\usetikzlibrary{arrows,snakes,backgrounds,patterns,matrix,shapes,fit,calc,shadows,plotmarks}
\usepackage[graphics,tightpage,active]{preview}
\PreviewEnvironment{tikzpicture}
\PreviewEnvironment{equation}
\PreviewEnvironment{equation*}
\newlength{\imagewidth}
\newlength{\imagescale}
\pagestyle{empty}
\thispagestyle{empty}
\begin{document}

\usetikzlibrary{positioning}
\tikzstyle{block} = [draw, fill=orange!30, rectangle, 
    minimum height=3em,rounded corners=3pt]
\tikzstyle{writereq} = [block]
\tikzstyle{output} = [blocke]

% The block diagram code is probably more verbose than necessary
\begin{tikzpicture}[auto, node distance=2cm,>=latex']



    % Entity chain
    \node [block] (V1-1) {Val1};
    \node [block, right=1mm of V1-1] (V2-1) {Val2};
    \node [block, right=1mm of V2-1] (V3-1) {Val2};
    \node [block, right=1mm of V3-1] (V4-1) {Val1};

    \node [draw, dashed, fit= (V1-1) (V2-1) (V3-1) (V4-1), inner sep=2mm] (Box1) {};


    \node [block, below=15mm of V2-1] (V1-2) {Val2};
    \node [block, right=1mm of V1-2] (V2-2) {Val1};

    \node [draw, dashed, fit= (V1-2) (V2-2), inner sep=2mm] (Box2) {};

    \node [block, below=15mm of V1-2] (V1-3) {Val1};
    \node [block, right=1mm of V1-3] (V2-3) {Val2};

    \node [draw, dashed, fit= (V1-3) (V2-3), inner sep=2mm] (Box3) {};

    \draw [->] (Box1) -- node {Create set} (Box2);
    \draw [->] (Box2) -- node {Sort elements} (Box3);
\end{tikzpicture}
\end{document}
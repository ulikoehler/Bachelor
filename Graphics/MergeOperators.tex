\documentclass{article}
\usepackage{tikz,amsmath,siunitx}
\usetikzlibrary{arrows,snakes,backgrounds,patterns,matrix,shapes,fit,calc,shadows,plotmarks}
\usepackage[graphics,tightpage,active]{preview}
\PreviewEnvironment{tikzpicture}
\PreviewEnvironment{equation}
\PreviewEnvironment{equation*}
\newlength{\imagewidth}
\newlength{\imagescale}
\pagestyle{empty}
\thispagestyle{empty}
\begin{document}

\usetikzlibrary{positioning}
\tikzstyle{block} = [draw, fill=orange!30, rectangle, 
    minimum height=3em, minimum width=6em, rounded corners=3pt]
\tikzstyle{writereq} = [block, ]
\tikzstyle{output} = [blocke]

% The block diagram code is probably more verbose than necessary
\begin{tikzpicture}[auto, node distance=2cm,>=latex']
    % Entity chain
    \node [writereq, align=center] (Write1) {\textbf{Write request}\\\textit{value 1}};
    \node [writereq, right=1cm of Write1, align=center] (Write2) {\textbf{Write request}\\\textit{value 2}};

    \node [writereq, yshift=-2cm, align=center] (MO) at ($(Write1)!0.5!(Write2)$) {\textbf{Merge operator}\\\texttt{Append}};

    \node [writereq, below=1cm of MO, align=center] (Result) {\textbf{Result}\\\textit{value1value2}};

    % Entity chain

    \draw [->] (Write1) -- (MO);
    \draw [->] (Write2) -- (MO);
    \draw [->] (MO) -- (Result);
\end{tikzpicture}
\end{document}